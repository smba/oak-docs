\section{Related Work} \label{sec:related_work}

\paragraph{Symbolic Execution}
Symbolic execution \cite{King1976,Darringer1978} remained an conceptual approach
for decades, but has been refined and extended over the last years. Several
approaches incorporating constraint solving and leveraging concrete execution
alongside symbolic execution emerged and enable automated test case generation;
the scalability of these approaches is still challenged by path explosion and constraint solving \cite{CadarSen2013}. 

Dynamic symbolic execution \cite{CadarSen2013} incorporates symbolic execution
alongside concrete execution and has shown promising results in automated test
generation\cite{artzi_finding_2008,artzi_finding_2010,DynamicWassermann} for PHP web
applications and might help mitigate the conceptual limitations faced with
symbolic values and dynamic features for future work on output-oriented
symbolic execution.

An overview about recent evolution of symbolic execution is provided by work of Cadar et al.
\cite{CadarSen2013}, as well as a more tool-centered perspective \cite{Cadar2011}.

\paragraph{Static Output Approximation}
To approximate program slices for web
applications, Ricca et al. \cite{tonella_web_2005,tonella_2001,tonella_2002} approximate
dynamically generated output. For output generating statements, such as \tt{echo} or
\tt{print}, all strings are unquoted (code extrusion). If those statements
contain variables, these are linked to string concatenations using a
proposed flow analysis called \emph{string-cat propagation}. From these
flows representing approximated output subsequent program slices are computed.

Minamide \cite{minamide_static_2005} approximates client page output
of web applications by describing possible output by a context-free
grammar that is constructed statically from the PHP code for a given regular
expression of user input. The constructed grammar enables analyses such as
detecting cross-site scripting vulnerabilities by checking whether user input
has been sanitized, and HTML validation by determining whether the constructed
grammar is contained in a depth-bound HTML grammar. Based on Minamide's
approximation approach several vulnerability analyses addressing cross-site
scripting \cite{wassermann_static_2008} and SQL injection
\cite{wassermann2007sound} have been proposed. Wang et al.
\cite{wang_locating_2010} utilizes this string analysis to detect strings
visible at the browser and enable internationalization of web applications.
    
Another approach is proposed by Wang et al. \cite{wang_automating_2012}, where
output for a web application is approximated using a hybrid approach: A
dynamic web page is executed with concrete input and the execution is
recorded at run-time. Changes in the client-side output then can be mapped to 
corresponding PHP code using static impact analysis.

\paragraph{Dynamic Features}
According to a case study by Hills et al. \cite{Hills:2013:ESP:2483760.2483786}
of the feature usage in PHP, based on a corpus of state-of-the-art-projects,
dynamic includes are less frequently used than static includes, yet usage
frequency varies from system to system. Further work by Hills et al.
\cite{hills2014static,hills2014php} approaches static resolution of dynamic includes using
both context-insensitive resolution on file level by simplifying PHP constants;
and context-sensitive on program level for transitive includes. In spite of 
promising results, these approaches, similar to our results, face limitations
for truly dynamic includes if resolution is not sound due to information not
being available in the source code like database query results.
