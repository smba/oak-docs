\section{Related Work} \label{sec:related_work}

\paragraph{Symbolic execution}
A literature review on symbolic execution is provided by Cadar in a
recent overview {\em et al.}~\cite{CadarSen2013}, as well as in a more
tool-centered perspective~\cite{Cadar2011}.
%
Symbolic execution \cite{King1976,Darringer1978} remained a
conceptual approach for decades, but has been refined and extended
over the last years. Several approaches incorporating constraint
solving and leveraging concrete execution alongside symbolic execution
emerged and enable automated test case generation; the scalability of
these approaches is still challenged by path explosion and constraint
solving.

\paragraph{Dynamic symbolic execution} Cedar and Sen~\cite{CadarSen2013} incorporates symbolic execution
alongside concrete execution and has shown promising results in
automated test
generation~\cite{artzi_finding_2008,artzi_finding_2010,DynamicWassermann}
for PHP web applications and might help mitigate the conceptual
limitations faced with symbolic values and dynamic features for future
work on output-oriented symbolic execution.

\paragraph{Symbolic Execution and Dynamic Analysis for PHP-based Web Applications}
PHPQuickFix and PHPRepair~\cite{ibm-icse12} detect and fix errors in
server-side PHP applications leading to ill-formed generated HTML.
PHPQuickFix examines {\em constant prints}, i.e., the PHP statements
that print directly string literals and repairs HTML ill-formed
errors, such as mismatched start and end tags and missing special
characters within those literals. It analyzes each string literal
separately and can only identify local issues. In contrast, PHPRepair
follows a dynamic approach in which a given test suite is used to
generate client-side code with different server-side executions, while
tracing the origin of output strings.
%% Final
They identify ill-formed HTML code, infer and automatically apply
fixes via using an external oracle and a string constraint solver.  In
contrast to their dynamic strategy, we aim to study symbolic execution
in this work.

Wang {\em et al.}~\cite{wang-fse12}'s approach executes the
(instrumented) server-side code with a specific input to generate the
client code. Changes in that generated code can then be mapped to
their origins in the PHP code using the recorded run-time mappings and
static impact analysis.
%We similarly track origin locations, but we symbolically execute the
%PHP code.
%
%We previously used our symbolic execution engine \emph{PhpSync} for
%similar purposes~\cite{drc-ase13,phpsync}. First, we 
PHPSync~\cite{phpsync} used static origin tracking to propagate
changes in the client code (output of symbolic execution) back to the
PHP code. Subsequently, the same authors designed DRC to analyze both
PHP code and generated client-side code to detect cross-language and
cross-stage dangling references~\cite{drc-ase13}. DRC matches path
constraints of references/declarations without fully symbolic
execution. It uses the path conditions from symbolic execution to
determine in which configurations dangling references can occur.

Apollo~\cite{artzi-icse10,artzi-tse10}, a more general fault
localization tool for PHP code, rates the PHP echo/print
statements higher in suspiciousness by running an instrumented
interpreter on test cases. Clark {\em et al.}~\cite{clark-ase11}
localize faults in an SQL command embedded within JSP code. 



\paragraph{Static Output Approximation}
To approximate program slices for web applications, Ricca et
al. \cite{tonella_web_2005,tonella_2001,tonella_2002} approximate
dynamically generated output. For output generating statements, such
as \tt{echo} or \tt{print}, all strings are unquoted (code
extrusion). If those statements contain variables, these are linked to
string concatenations using a proposed flow analysis called
\emph{string-cat propagation}. From these flows representing
approximated output subsequent program slices are computed.

Minamide \cite{minamide_static_2005} approximates client page output
of web applications by describing possible output by a context-free
grammar that is constructed statically from the PHP code for a given regular
expression of user input. The constructed grammar enables analyses such as
detecting cross-site scripting vulnerabilities by checking whether user input
has been sanitized, and HTML validation by determining whether the constructed
grammar is contained in a depth-bound HTML grammar. Based on Minamide's
approximation approach several vulnerability analyses addressing cross-site
scripting \cite{wassermann_static_2008} and SQL injection
\cite{wassermann2007sound} have been proposed. Wang et al.
\cite{wang_locating_2010} utilizes this string analysis to detect strings
visible at the browser and enable internationalization of web applications.
    
Another approach is proposed by Wang et al. \cite{wang_automating_2012}, where
output for a web application is approximated using a hybrid approach: A
dynamic web page is executed with concrete input and the execution is
recorded at run-time. Changes in the client-side output then can be mapped to 
corresponding PHP code using static impact analysis.

%Minamide proposed a string analyzer~\cite{minamide05} that takes a PHP
%program and a regular expression describing the input, and validates
%approximate HTML output via context-free grammar analysis.
%regular expression describing all of its possible inputs, and then
%statically approximates and validates the output via a context-free
%grammar. In comparison, his goal is to validate approximated HTML
%outputs from a PHP program {\em without fixing support}. Moreover,
%{\tool} performs symbolic execution
%to produce D-models without
%requiring an input specification as in Minamide's.
%%Using a similar string analyzer,
%Wang {\em et al.}~\cite{wang-fse10} compute the approximated output of
%PHP code and identify the constant strings visible from a browser
%for translation. Both do not aim to analyze multiple variants of
%embedded code in JS or CSS.
% getting Telvik Bultan work

Several string taint-analysis techniques were built for PHP web
programs and software-security
problems~\cite{kieyzun09,su-icse08,xieaiken06,tevfik11}.  The key idea
behind string tainting is to check the code expression by expression,
variable by variable to have a list of all the ones potentially
influenced by outside input.  If these variables are used in running a
dangerous command, we will have potentially dangerous tainted
variables.
%
Gould {\em et al.}~\cite{gould04} use string analysis to guarantee
well-typed SQL queries generated by a Java program. The type system
in~\cite{tabuchi03} is based on regular expressions with string
concatenation and pattern matching.  

%A CFG-based type system for string analysis is presented
%in~\cite{thiemann05}.

\paragraph{Dynamic Features}
According to a case study by Hills et al. \cite{Hills:2013:ESP:2483760.2483786}
of the feature usage in PHP, based on a corpus of state-of-the-art-projects,
dynamic includes are less frequently used than static includes, yet usage
frequency varies from system to system. Further work by Hills et al.
\cite{hills2014static,hills2014php} approaches static resolution of dynamic includes using
both context-insensitive resolution on file level by simplifying PHP constants;
and context-sensitive on program level for transitive includes. In spite of 
promising results, these approaches, similar to our results, face limitations
for truly dynamic includes if resolution is not sound due to information not
being available in the source code like database query results.
